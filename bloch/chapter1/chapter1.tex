\documentclass{article}
\usepackage[utf8]{inputenc}
\usepackage[a4paper, total={6in, 8in}]{geometry}
\usepackage{amsmath, amssymb, amsthm}

\newtheorem{theorem}{Theorem}

\newtheorem*{lemma}{Lemma}
\newtheorem*{definition}{Definition}
\newtheorem*{corollary}{Corollary}
\newtheorem*{axiom}{Axiom}

\newcommand{\N}{\mathbb{N}}
\newcommand{\Z}{\mathbb{Z}}
\newcommand{\Q}{\mathbb{Q}}
\newcommand{\R}{\mathbb{R}}
\newcommand{\C}{\mathbb{C}}

\begin{document}

\section{Axioms for the natural numbers}

\begin{axiom}
	\textbf{(Peano Postulates).} There exists a set $\N$ with an element $1 \in \N$ and a function $s: \N \rightarrow \N$ that satisfy the following three properties.
	\begin{enumerate}
		\item[a.] There is no $n \in \N$ such that $s(n) = 1$.
		\item[b.] The function is injective.
		\item[c.] Let $G \subseteq \N$ be a set. Suppose that $1 \in G$ and that if $g \in G$ then $s(g) \in G$. Then $G = \N$.
	\end{enumerate}
\end{axiom}

\begin{definition}
	The set of \textbf{natural numbers}, denoted $\N$, is the set the existence of which is given in the Peano Postulates.
\end{definition}

\begin{theorem}
	\textbf{(Definition by Recursion).} Let $H$ be a set, and let $e \in H$ and let $k: H \rightarrow H$ be a function. Then there is a unique function $f: \N \rightarrow H$ such that $f(1) = e$ and that $f \circ s = k \circ f$.
\end{theorem}

\begin{theorem}
	There is a unique binary operation $+: \N \times \N \rightarrow \N$ that satisfies the following two properties for all $n,m \in \N$.
	\begin{enumerate}
		\item[a.] $n + 1 = s(n)$
		\item[b.] $n + s(m) = s(n + m)$
	\end{enumerate}
\end{theorem}

\begin{theorem}
	There is a unique binary operation $\cdot : \N \times \N \rightarrow \N$ that satisfies the following two properties for all $n,m \in \N$.
	\begin{enumerate}
		\item[a.] $n \cdot 1 = n$
		\item[b.] $n \cdot s(m) = (n \cdot m) + n$
	\end{enumerate}
\end{theorem}

\begin{theorem}
	Let $a, b, c \in \N$.
	\begin{enumerate}
		\item If $a + c = b + c$ then $a = b$ (Cancellation Law for Addition).
		\item $(a + b) + c = a + (b + c)$ (Associative Law for Addition).
		\item $1 + a = s(a) = a + 1$.
		\item $a + b = b + a$ (Commutative Law for Addition).
		\item $a + b \ne 1$.
		\item $a + b \ne a$.
		\item $a \cdot 1 = a = 1\cdot a$ (Identity Law for Multiplication).
		\item $(a + b)c = ac + bc$ (Distributive Law).
		\item $ab = ba$ (Commutative Law for Multiplication).
		\item $(ab)c = a(bc)$ (Associative Law for Multiplication).
		\item If $ac = bc$ then $a = b$ (Cancellation Law for Multiplication).
		\item $ab = 1$ if and only if $a = 1 = b$.
	\end{enumerate}
\end{theorem}

\begin{definition}
	The relation $<$ on $\N$ is defined by $a < b$ if and only if there is some $p \in \N$ such that $a + p = b$ for all $a,b \in \N$. The relation $\le$ on $\N$ is defined by $a \le b$ if and only if $a < b$ or $a = b$, for all $a, b \in \N$.
\end{definition}

\begin{theorem}
	Let $a,b,c,d \in \N$.
	\begin{enumerate}
		\item $a \le a$ and $a \nless a$, and $a < a + 1$.
		\item $1 \le a$.
		\item If $a < b$ and $b < c$, then $a < c$; if $a \le b$ and $b < c$ then $a < c$; if $a < b$ and $b \le c$ then $a < c$; if $a \le b$ and $b \le c$ then $a \le c$.
		\item $a < b$ if and only if $a + c < b + c$.
		\item $a < b$ if and only if $ac < bc$.
		\item Precisely one of $a < b$ or $a = b$ or $a > b$ holds (Trichotomy Law).
		\item $a \le b$ or $b \le a$
		\item If $a \le b$ and $b \le a$ then $a = b$.
		\item It cannot be that $b < a < b + 1$.
		\item $a \le b$ if and only if $a < b + 1$.
		\item $a < b$ if and only if $a + 1 \le b$.
	\end{enumerate}
\end{theorem}

\begin{theorem}
	\textbf{(Well-Ordering Principle).} Let $G \subseteq \N$ be a non-empty set. Then there is some $m \in G$ such that $m \le g$ for all $g \in G$.
\end{theorem}

\subsection*{Exercises}

\subsubsection*{1. Fill in the missing details in the proof of Theorem 1.2.6.}

\end{document}