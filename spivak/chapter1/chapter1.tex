\documentclass{article}
\usepackage[utf8]{inputenc}
\usepackage[a4paper, total={6in, 8in}]{geometry}
\usepackage{amsmath, amssymb, amsthm}

\newtheorem{theorem}{Theorem}

\newtheorem*{definition}{Definition}
\newtheorem*{corollary}{Corollary}

\begin{document}

\subsubsection*{19. The fact that $a^2 \ge 0$ for all numbers $a$, elementary as it may seem, is nevertheless the fundamental idea upon which most important inequalities are ultimately based. The great-granddaddy of all inequalities is the \textit{Schwarz inequality}: \[x_1y_1 + x_2y_2 \le \sqrt{{x_1}^2 + {x_2}^2}\sqrt{{y_1}^2 + {y_2}^2}.\] The three proofs of the Schwarz inequality outlined below have only one thing in common - their reliance on the fact that $a^2 \ge 0$ for all $a$.}
\begin{enumerate}
	\item[(a)] Prove that if $x_1 = \lambda y_1$ and $x_2 = \lambda y_2$ for some number $\lambda \ge 0$, then equality holds in the Schwarz inequality. Prove the same thing if $y_1 = y_2 = 0$. Now suppose that $y_1$ and $y_2$ are not both 0, and that there is no number $\lambda$ such that $x_1 = \lambda y_1$ and $x_2 = \lambda y_2$. Then \[0 < (\lambda y_1 - x_1)^2 + (\lambda y_2 - x_2)^2 = \lambda^2(y_1^2 + y_2^2) - 2\lambda (x_1y_1 + x_2y_2) + (x_1^2 + x_2^2)\] Using problem 18, complete the proof of the Schwarz inequality.
	\begin{proof}
		Suppose that for some number $\lambda \ge 0$, $x_1 = \lambda y_1$ and $x_2 = \lambda y_2$. Then
		\begin{align*}
			\sqrt{x_1^2 + x_2^2}\sqrt{y_1^2 + y_2^2} &= \sqrt{x_1^2 + x_2^2}\sqrt{(\lambda x_1)^2 + (\lambda x_2)^2} = \sqrt{x_1^2 + x_2^2}\sqrt{\lambda^2(x_1^2 + x_2^2)} \\&= \lambda (\sqrt{x_1^2 + x_2^2})^2 = \lambda(x_1^2 + x_2^2) = \lambda x_1^2 + \lambda x_2^2 \\&= x_1\lambda x_1 + x_2\lambda x_2 = x_1y_1 + x_2y_2,
		\end{align*}
		so equality holds.
		\\Next, suppose that $y_1 = y_2 = 0$. Then
		\begin{align*}
			\sqrt{x_1^2 + x_2^2}\sqrt{y_1^2 + y_2^2} = \sqrt{x_1^2 + x_2^2}\sqrt{0} = 0\sqrt{x_1^2 + x_2^2} = 0 = 0x_1 + 0x_2 = x_1y_1 + x_2y_2.
		\end{align*}
		Finally, suppose that $y_1$ and $y_2$ are not both 0, and that there is no number $\lambda$ such that $x_1 = \lambda y_1$ and $x_2 = \lambda y_2$. Then the equation $\lambda^2(y_1^2 + y_2^2) - 2\lambda(x_1y_1 + x_2y_2) + (x_1^2 + x_2^2) = 0$ has no solution $\lambda$. We can then divide by $(y_1^2 + y_2^2)$ to obtain $\lambda^2 + (\frac{-2(x_1y_1 + x_2y_2)}{(y_1^2 + y_2^2)})\lambda + \frac{(x_1^2 + x_2^2)}{(y_1^2 + y_2^2)} = 0$. Then if $b = \frac{-2(x_1y_1 + x_2y_2)}{y_1^2 + y_2^2}$ and $c = \frac{x_1^2 + x_2^2}{y_1^2 + y_2^2}$, from problem 18, we have $b^2 - 4c < 0$. Then \[ b^2 - 4c = (\frac{-2(x_1y_1 + x_2y_2)}{y_1^2 + y_2^2})^2 - \frac{4(x_1^2 + x_2^2)}{y_1^2 + y_2^2} = \frac{4(x_1y_1 + x_2y_2)^2}{(y_1^2 + y_2^2)^2} - \frac{4(x_1^2 + x_2^2)}{y_1^2 + y_2^2}, \] so \[(x_1y_1 + x_2y_2)^2 - (x_1^2 + x_2^2)(y_1^2 + y_2^2) < 0,\] which finally results in \[(x_1y_1 + x_2y_2) < \sqrt{x_1^2 + x_2^2}\sqrt{y_1^2 + y_2^2}.\]
	\end{proof}
	\pagebreak
	\item[(b)] Prove the Schwarz inequality by using $2xy \le x^2 + y^2$ with $x = \frac{x_i}{\sqrt{{x_1}^2 + {x_2}^2}}$ and $y = \frac{y_i}{\sqrt{{y_1}^2 + {y_2}^2}}$ first for $i = 1$ and then for $i = 2$.
	\begin{proof}
		Let $x = \frac{x_i}{\sqrt{{x_1}^2 + {x_2}^2}}$ and $y = \frac{y_i}{\sqrt{{y_1}^2 + {y_2}^2}}$ for $i=1$. Then \[2xy = 2 \frac{x_1y_1}{\sqrt{{x_1}^2 + {x_2}^2}\sqrt{{y_1}^2 + {y_2^2}}} \text{ and } x^2 + y^2 = \frac{{x_1}^2}{{x_1}^2 + {x_2}^2} + \frac{{y_1}^2}{{y_1}^2 + {y_2}^2}.\] Next, for $i=2$ \[2xy = 2 \frac{x_2y_2}{\sqrt{{x_1}^2 + {x_2}^2}\sqrt{{y_1}^2 + {y_2^2}}} \text{ and } x^2 + y^2 = \frac{{x_2}^2}{{x_1}^2 + {x_2}^2} + \frac{{y_2}^2}{{y_1}^2 + {y_2}^2}.\] Knowing that $2xy \le x^2 + y^2$, derived from $(x - y)^2 \ge 0$, we can sum the inequalities to obtain \[ 2(\frac{x_1y_1 + x_2y_2}{\sqrt{{x_1}^2 + {x_2}^2}\sqrt{{y_1}^2 + {y_2}^2}}) \le \frac{{x_1}^2 + {x_2}^2}{{x_1}^2 + {x_2}^2} + \frac{{y_1}^2 + {y_2}^2}{{y_1}^2 + {y_2}^2} = 2, \] so \[ \frac{x_1y_1 + x_2y_2}{\sqrt{{x_1}^2 + {x_2}^2}\sqrt{{y_1}^2 + {y_2}^2}} \le 1 \text{ and } x_1y_1 + x_2y_2 \le \sqrt{{x_1}^2 + {x_2}^2}\sqrt{{y_1}^2 + {y_2}^2}.\]
	\end{proof}
	\item[(c)] Prove the Schwarz inequality by first proving that \[ ({x_1}^2 + {x_2}^2)({y_1}^2 + {y_2}^2) = (x_1y_1 + x_2y_2)^2 + (x_1y_2 - x_2y_1)^2. \]
	\begin{proof}
		For some numbers $x_1, x_2, y_1, y_2$, we have
		\begin{align*}
			(x_1y_1 + x_2y_2)^2 + (x_1y_2 - x_2y_1)^2 &= {x_1}^2{y_1}^2 + 2x_1x_2y_1y_2 + {x_2}^2{y_2}^2 + {x_1}^2{y_2}^2 - 2x_1x_2y_1y_2 + {x_2}^2{y_1}^2 \\&= {x_1}^2{y_1}^2 + {x_2}^2{y_1}^2 + {x_1}^2{y_2}^2 + {x_2}^2{y_2}^2 \\&= ({x_1}^2 + {x_2}^2)({y_1}^2 + {y_2}^2).
		\end{align*}
		Then since $(x_1y_2 - x_2y_1)^2 \ge 0$, we have \[ (x_1y_1 + x_2y_2)^2 \le ({x_1}^2 + {x_2}^2)({y_1}^2 + {y_2}^2) \text{ and } x_1y_2 + x_2y_2 \le \sqrt{{x_1}^2 + {x_2}^2}\sqrt{{y_1}^2 + {y_2}^2}. \]
	\end{proof}
	\item[(d)] Deduce, from each of these three proofs, that equality holds only when $y_1 = y_2 = 0$ or when there is a number $\lambda \ge 0$ such that $x_1 = \lambda y_1$ and $x_2 = \lambda y_2$.
	\begin{proof}
		For proof (a), it is already shown that if $y_1$ and $y_2$ are not both 0, and there is no number $\lambda \ge 0$ such that $x_1 = \lambda y_1$ and $x_2 = \lambda y_2$, then $x_1y_1 + x_2y_2 < \sqrt{{x_1}^2 + {x_2}^2}\sqrt{{y_1}^2 + {y_2}^2}$.
		For proof (b) we used the fact that $(x - y)^2 \ge 0$. Equality then holds when $x = y$, so \[ \frac{x_i}{\sqrt{{x_1}^2 + {x_2}^2}} = \frac{y_i}{\sqrt{{y_1}^2 + {y_2}^2}} \text{ and } x_i = y_i\frac{\sqrt{{x_1}^2 + {x_2}^2}}{\sqrt{{y_1}^2 + {y_2}^2}}, \] so $x_i = \lambda y_i$.

		For proof (c), we've shown that equality holds when $(x_1y_2 - x_2y_1)^2 = 0$, so when $x_1y_2 = x_2y_1$. Then either $y_1 = y_2 = 0$, or if we assume without loss of generality that $y_2 \ne 0$, we can take $\lambda = \frac{x_2}{y_2}$ so $x_1 = \lambda y_1$ and $x_2 = \lambda y_2$.
	\end{proof}
\end{enumerate}

\subsubsection*{20. Prove that if \[ |x - x_0| < \frac{\epsilon}{2} \text{ and } |y-y_0| < \frac{\epsilon}{2}, \] then \[ |(x + y) - (x_0 + y_0)| < \epsilon \text{ and } |(x - y) - (x_0 - y_0)| < \epsilon. \]}
\begin{proof}
	Suppose that $|x - x_0| < \frac{\epsilon}{2}$ and $|y - y_0| < \frac{\epsilon}{2}$. Then \[ |(x + y) - (x_0 + y_0)| = |(x - x_0) + (y - y_0)| \le |x - x_0| + |y - y_0| < \frac{\epsilon}{2} + \frac{\epsilon}{2} = \epsilon, \] and \[ |(x - y) - (x_0 - y_0)| = |(x - x_0) + (y_0 - y)| \le |x - x_0| + |y_0 - y| = |x - x_0| + |y - y_0| = \frac{\epsilon}{2} + \frac{\epsilon}{2} = \epsilon \]
\end{proof}

\subsubsection*{21. Prove that if \[ |x - x_0| < \text{min}(\frac{\epsilon}{2(|y_0| + 1)}, 1) \text{ and } |y - y_0| < \frac{\epsilon}{2(|x_0| + 1)}, \] then $|xy - x_0y_0| < \epsilon$.}
\begin{proof}
	We have $|x| - |x_0| \le |x - x_0| < 1$, so $|x| < 1 + |x_0|$. Then
	\begin{align*}
		|xy - x_0y_0| = |x(y - y_0) + y_0(x - x_0)| \le |x||y - y_0| + |y_0||x - x_0| &< (1 + |x_0|)|y - y_0| + |y_0||x - x_0| \\&= \frac{\epsilon}{2} + |y_0||x - x_0|,
	\end{align*}
	and since $|x - x_0| < \frac{\epsilon}{2(|y_0| + 1)}$, then \[ |x - x_0|(|y_0| + 1) = |y_0||x-x_0| + |x -x_0| < \frac{\epsilon}{2}, \] so $\frac{\epsilon}{2} + |y_0||x - x_0| < \epsilon$, therefore $|xy - x_0y_0| < \epsilon$. 
\end{proof}

\subsubsection*{22. Prove that if $y_0 \ne 0$ and \[ |y - y_0| < \text{min}(\frac{|y_0|}{2}, \frac{\epsilon|y_0|^2}{2}), \] then $y \ne 0$ and \[ |\frac{1}{y} - \frac{1}{y_0}| < \epsilon. \]}
\begin{proof}
	We have $|y_0| = |(y_0 - y) + y| \le |y - y_0| + |y|$, so since $|y - y_0| < \frac{|y_0|}{2}$, then \[ |y| \ge |y_0| - |y - y_0| > |y_0| - \frac{|y_0|}{2} = \frac{|y_0|}{2}. \] Then \[ \left|\frac{1}{y} - \frac{1}{y_0}\right| = \left|\frac{y_0 - y}{yy_0}\right| = \frac{|y - y_0|}{|y||y_0|} < \frac{\frac{\epsilon|y_0|^2}{2}}{|y||y_0|} < \frac{\frac{\epsilon|y_0|^2}{2}}{\frac{|y_0|}{2}|y_0|} = \epsilon \]
\end{proof}
\pagebreak
\subsubsection*{23. Replace the question marks in the following statement by expressions involving $\epsilon, x_0$ and $y_0$, so that the conclusion will be true:}
If $y_0 \ne 0$ and \[ \left|y - y_0\right| < ? \text{ and } \left|x - x_0\right|< ? \] then $y \ne 0$ and \[ \left|\frac{x}{y} - \frac{x_0}{y_0}\right| < \epsilon. \]
\begin{proof}
	We've shown in problem 21, that if $|x - x_0| < \text{min}(\frac{\epsilon}{2(|y_0| + 1)}, 1)$ and $|y - y_0| < \frac{\epsilon}{2(|x_0| + 1)}$, then $|xy - x_0y_0| < \epsilon$. Then for $\left|\frac{x}{y} - \frac{x_0}{y_0}\right| < \epsilon$ to be true, we want $|x - x_0| < \text{min}(\frac{\epsilon}{2(|y_0| + 1)}, 1)$ and we need to find a condition for $|y - y_0| < ?$ such that $|\frac{1}{y} - \frac{1}{y_0}| < \frac{\epsilon}{2(|x_0| + 1)}$. We can then substitute $\epsilon$ in problem 22 for $\frac{\epsilon}{2(|x_0| + 1)}$, so \[ |y - y_0| < min(\frac{|y_0|}{2}, \frac{\epsilon|y_0|^2}{4(|x_0| + 1)}). \]
\end{proof}

\end{document}	