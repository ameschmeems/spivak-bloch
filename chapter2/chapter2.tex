\documentclass{article}
\usepackage[utf8]{inputenc}
\usepackage[a4paper, total={6in, 8in}]{geometry}
\usepackage{amsmath, amssymb, amsthm}

\newtheorem{theorem}{Theorem}

\newtheorem*{definition}{Definition}
\newtheorem*{corollary}{Corollary}

\begin{document}

\section*{Numbers of various sorts}

\subsection*{Exercises}

\subsubsection*{1. Prove the following formulas by induction.}
\begin{enumerate}
	\item[(i)] $1^2 + \dots + n^2 = \frac{n(n + 1)(2n + 1)}{6}$
	\begin{proof}
		Let $n = 1$. Then $\frac{1(2)(3)}{6} = 1$, so the formula holds. Now assume that the formula is true for some $k \in N$. Then 
		\begin{align*}	
			1^2 + \dots + (k + 1)^2 &= 1^2 + \dots + k^2 + (k + 1)^2 = \frac{k(k + 1)(2k + 1)}{6} + (k + 1)^2 \\&= \frac{k(k + 1)(2k + 1) + 6(k + 1)^2}{6} = \frac{2k^3 + 9k^2 + 13k + 6}{6} \\&= \frac{(k + 1)(k + 2)(2(k + 1) + 1)}{6}
		\end{align*}
	\end{proof}
	\item[(ii)] $1^3 + \dots + n^3 = (1 + \dots + n)^2$
	\begin{proof}
		Let $n = 1$. Then $1^3 = 1^2$, so the formula holds. Now assume that the formula is true for some $k \in N$. Then
		\begin{align*}
			1^3 + \dots + (k + 1)^3 &= (1^3 + \dots + k^3) + (k + 1)^3 = (1 + \dots + k)^2 + (k + 1)^3 \\&= (1 + \dots + k)^2 + (k + 1)^2(k + 1) = (1 + \dots + k)^2 + k(k + 1)^2 + (k + 1)^2 \\&= (1 + \dots + k)^2 + 2\frac{k(k + 1)}{2}(k + 1) + (k + 1)^2 \\&= (1 + \dots + k)^2 + 2(1 + \dots + k)(k + 1) + (k + 1)^2 \\&= (1 + \dots + (k + 1))^2
		\end{align*}
	\end{proof}
\end{enumerate}

\subsubsection*{2. Find a formula for}
\begin{enumerate}
	\item[(i)] $\sum^{n}_{i = 1}(2i - 1) = 1 + 3 + 5 + \dots + (2n - 1)$
	\begin{proof}
		\begin{align*}
			\sum^{n}_{i = 1}(2i - 1) &= 1 + 2 + \dots + 2n - 2(1 + 2 + \dots + n) = \frac{2n(2n + 1)}{2} - 2\frac{n(n+1)}{2} \\&= n(2n + 1) - n(n + 1) = 2n^2 + n - n^2 - n = n^2.
		\end{align*}
	\end{proof}
	\pagebreak
	\item[(ii)] $\sum^{n}_{i = 1}(2i - 1)^2 = 1^2 + 3^2 + 5^2 + \dots + (2n - 1)^2$
	\begin{proof}
		\begin{align*}
			\sum^{n}_{i = 1}(2i - 1)^2 &= 1^2 + 2^2 + \dots + (2n)^2 - 4(1^2 + 2^2 + \dots + n^2) = \frac{2n(2n + 1)(4n + 1)}{6} - 4\frac{n(n+1)(2n+1)}{6} \\&= \frac{8n^3 - 2n}{6} = \frac{2n(2n - 1)(2n + 1)}{6}
		\end{align*}
	\end{proof}
\end{enumerate}

\subsubsection*{3. If $0 \le k \le n$, the "binomial coefficient" $\binom{n}{k}$ is defined by \[ \binom{n}{k} = \frac{n!}{k!(n - k)!} = \frac{n(n - 1)\dots(n - k + 1)}{k!}\text{, if }k \ne 0, n \] \[ \binom{n}{0} = \binom{n}{n} = 1\text{ (a special case of the first formula if we define 0! = 1)}, \] and for $k < 0$ or $k > n$ we just define the binomial coefficient to be 0.}
\begin{enumerate}
	\item[(a)] Prove that \[ \binom{n + 1}{k} = \binom{n}{k - 1} + \binom{n}{k} \]
	
	\begin{proof}
		\begin{align*}
			\binom{n}{k - 1} + \binom{n}{k} &= \frac{n!}{(k - 1)!(n - (k - 1))!} + \frac{n!}{k!(n - k)!} = \frac{kn!}{k!(n + 1 - k)!} + \frac{(n + 1 - k)n!}{k!(n + 1 - k)!} \\&= \frac{n!(k + n + 1 - k)}{k!(n + 1 - k)!} = \frac{(n + 1)!}{k!(n + 1 - k)!} = \binom{n + 1}{k}
		\end{align*}
	\end{proof}
	\item[(b)] Notice that all the numbers in Pascal's triangle are natural numbers. Use part (a) to prove by induction that $\binom{n}{k}$ is always a natural number.
	
	\begin{proof}
		Let $n = 1$. Then $\binom{1}{0} = 1$ and $\binom{1}{1} = 1$, so the binomial coefficient is always a natural number. Next, suppose that for some number $n$, and $0 \le k \le n$, $\binom{n}{k}$ is always a natural number. Then if $k = 0$ or $k = n + 1$, then $\binom{n + 1}{k} = 1$, which is a natural number. Otherwise, \[ \binom{n + 1}{k} = \binom{n}{k} + \binom{n}{k - 1}\text{, and } 1 \le k \le n, 0 \le k - 1 \le n - 1, \] so $\binom{n + 1}{k}$ is a sum of two natural numbers, and is therefore also a natural number.
	\end{proof}
	\item[(c)] Give another proof that $\binom{n}{k}$ is a natural number by showing that $\binom{n}{k}$ is the number of sets of exactly k integers chosen from $1, \dots, n$.
\end{enumerate}

\end{document}