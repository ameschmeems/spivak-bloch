\documentclass{article}
\usepackage[utf8]{inputenc}
\usepackage[a4paper, total={6in, 8in}]{geometry}
\usepackage{amsmath, amssymb, amsthm}

\newtheorem{theorem}{Theorem}

\newtheorem*{definition}{Definition}
\newtheorem*{corollary}{Corollary}

\begin{document}

\section*{Numbers of various sorts}

\subsection*{Exercises}

\subsubsection*{1. Prove the following formulas by induction.}
\begin{enumerate}
	\item[(i)] $1^2 + \dots + n^2 = \frac{n(n + 1)(2n + 1)}{6}$
	\begin{proof}
		Let $n = 1$. Then $\frac{1(2)(3)}{6} = 1$, so the formula holds. Now assume that the formula is true for some $k \in N$. Then 
		\begin{align*}	
			1^2 + \dots + (k + 1)^2 &= 1^2 + \dots + k^2 + (k + 1)^2 = \frac{k(k + 1)(2k + 1)}{6} + (k + 1)^2 \\&= \frac{k(k + 1)(2k + 1) + 6(k + 1)^2}{6} = \frac{2k^3 + 9k^2 + 13k + 6}{6} \\&= \frac{(k + 1)(k + 2)(2(k + 1) + 1)}{6}
		\end{align*}
	\end{proof}
	\item[(ii)] $1^3 + \dots + n^3 = (1 + \dots + n)^2$
	\begin{proof}
		Let $n = 1$. Then $1^3 = 1^2$, so the formula holds. Now assume that the formula is true for some $k \in N$. Then
		\begin{align*}
			1^3 + \dots + (k + 1)^3 &= (1^3 + \dots + k^3) + (k + 1)^3 = (1 + \dots + k)^2 + (k + 1)^3 \\&= (1 + \dots + k)^2 + (k + 1)^2(k + 1) = (1 + \dots + k)^2 + k(k + 1)^2 + (k + 1)^2 \\&= (1 + \dots + k)^2 + 2\frac{k(k + 1)}{2}(k + 1) + (k + 1)^2 \\&= (1 + \dots + k)^2 + 2(1 + \dots + k)(k + 1) + (k + 1)^2 \\&= (1 + \dots + (k + 1))^2
		\end{align*}
	\end{proof}
\end{enumerate}

\subsubsection*{2. Find a formula for}
\begin{enumerate}
	\item[(i)] $\sum^{n}_{i = 1}(2i - 1) = 1 + 3 + 5 + \dots + (2n - 1)$
	\begin{proof}
		\begin{align*}
			\sum^{n}_{i = 1}(2i - 1) &= 1 + 2 + \dots + 2n - 2(1 + 2 + \dots + n) = \frac{2n(2n + 1)}{2} - 2\frac{n(n+1)}{2} \\&= n(2n + 1) - n(n + 1) = 2n^2 + n - n^2 - n = n^2.
		\end{align*}
	\end{proof}
	\pagebreak
	\item[(ii)] $\sum^{n}_{i = 1}(2i - 1)^2 = 1^2 + 3^2 + 5^2 + \dots + (2n - 1)^2$
	\begin{proof}
		\begin{align*}
			\sum^{n}_{i = 1}(2i - 1)^2 &= 1^2 + 2^2 + \dots + (2n)^2 - 4(1^2 + 2^2 + \dots + n^2) = \frac{2n(2n + 1)(4n + 1)}{6} - 4\frac{n(n+1)(2n+1)}{6} \\&= \frac{8n^3 - 2n}{6} = \frac{2n(2n - 1)(2n + 1)}{6}
		\end{align*}
	\end{proof}
\end{enumerate}

\subsubsection*{3. If $0 \le k \le n$, the "binomial coefficient" $\binom{n}{k}$ is defined by \[ \binom{n}{k} = \frac{n!}{k!(n - k)!} = \frac{n(n - 1)\dots(n - k + 1)}{k!}\text{, if }k \ne 0, n \] \[ \binom{n}{0} = \binom{n}{n} = 1\text{ (a special case of the first formula if we define 0! = 1)}, \] and for $k < 0$ or $k > n$ we just define the binomial coefficient to be 0.}
\begin{enumerate}
	\item[(a)] Prove that \[ \binom{n + 1}{k} = \binom{n}{k - 1} + \binom{n}{k} \]
	
	\begin{proof}
		\begin{align*}
			\binom{n}{k - 1} + \binom{n}{k} &= \frac{n!}{(k - 1)!(n - (k - 1))!} + \frac{n!}{k!(n - k)!} = \frac{kn!}{k!(n + 1 - k)!} + \frac{(n + 1 - k)n!}{k!(n + 1 - k)!} \\&= \frac{n!(k + n + 1 - k)}{k!(n + 1 - k)!} = \frac{(n + 1)!}{k!(n + 1 - k)!} = \binom{n + 1}{k}
		\end{align*}
	\end{proof}
	\item[(b)] Notice that all the numbers in Pascal's triangle are natural numbers. Use part (a) to prove by induction that $\binom{n}{k}$ is always a natural number.
	
	\begin{proof}
		Let $n = 1$. Then $\binom{1}{0} = 1$ and $\binom{1}{1} = 1$, so the binomial coefficient is always a natural number. Next, suppose that for some number $n$, and $0 \le k \le n$, $\binom{n}{k}$ is always a natural number. Then if $k = 0$ or $k = n + 1$, then $\binom{n + 1}{k} = 1$, which is a natural number. Otherwise, \[ \binom{n + 1}{k} = \binom{n}{k} + \binom{n}{k - 1}\text{, and } 1 \le k \le n, 0 \le k - 1 \le n - 1, \] so $\binom{n + 1}{k}$ is a sum of two natural numbers, and is therefore also a natural number.
	\end{proof}
	\pagebreak
	\item[(c)] Give another proof that $\binom{n}{k}$ is a natural number by showing that $\binom{n}{k}$ is the number of sets of exactly k integers chosen from $1, \dots, n$.
	
	\begin{proof}
		The number of $k$-tuples of integers chosen from $1, \dots, n$ is $n(n - 1)\dots(n - k + 1)$, because there is $n$ choices for the first element, $n - 1$ choices for the second, etc. Now, for each $k$-tuple, it can be arranged in $k(k - 1)\dots(1) = k!$ different ways, so to get the number of sets of size $k$, with elements chosen from $1, \dots, n$, we have $\frac{n(n - 1)\dots(n - k + 1)}{k!} = \binom{n}{k}$.
	\end{proof}
	\item[(d)] Prove the "binomial theorem": If $a$ and $b$ are any numbers and $n$ is a natural number, then 
	\begin{align*}
		(a + b)^n &= a^n + \binom{n}{1}a^{n - 1}b + \binom{n}{2}a^{n-2}b^2 + \dots + \binom{n}{n - 1}ab^{n - 1} + b^n \\&=\sum_{j = 0}^{n}\binom{n}{j}a^{n - j}b^j.
	\end{align*}
	\begin{proof}
		Let $n = 1$. Then \[(a + b)^1 = a + b = a^1 + b^1 = \sum_{j = 0}^{1}\binom{n}{j}a^{n - j}b^j,\] so the statement holds true.

		Next, suppose that the statement is true for some $n \ge 1$. Then
		\begin{align*}
			(a + b)^{n + 1} &= (a + b)(a + b)^n = (a + b)\sum_{j = 0}^{n}\binom{n}{j}a^{n - j}b^j \\&= \sum_{j=0}^{n}\binom{n}{j}a^{n + 1 - j}b^j + \sum_{j=0}^{n}\binom{n}{j}a^{n - j}b^{j + 1} \\&= \sum_{j=0}^{n}\binom{n}{j}a^{n + 1 - j}b^j + \sum_{j = 1}^{n + 1}\binom{n}{j - 1}a^{n + 1 - j}b^j \\&= a^{n+1} + \sum_{j=1}^{n}(\binom{n}{j} + \binom{n}{j - 1})a^{n+1-j}b^j + b^{n + 1} \\&= a^{n+1} + \sum_{j=1}^{n}\binom{n+1}{j}a^{n+1-j}b^j + b^{n+1} \\&= \sum_{j=0}^{n+1}a^{n+1-j}b^j.
		\end{align*}
	\end{proof}
	\pagebreak
	\item[(e)] Prove that
	\begin{enumerate}
		\item[(i)] \[ \sum_{j=0}^{n}\binom{n}{j} = \binom{n}{0} + \dots + \binom{n}{n} = 2^n \]
		\begin{proof}
			Let $n = 1$. Then \[ \sum_{j = 0}^{n}\binom{n}{j} = 1 + 1 = 2 = 2^1, \]so the formula holds.

			Next, suppose that for some $n \ge 1$, $\sum_{j=0}^{n} = 2^n$. Then
			\begin{align*}
				\sum_{j=0}^{n+1}\binom{n+1}{j} &= \sum_{j=0}^{n+1}\left(\binom{n}{j} + \binom{n}{j-1}\right) = \binom{n}{0} + \sum_{j=1}^{n}\binom{n}{j} + \sum_{j=1}^{n}\binom{n}{j - 1} + \binom{n}{n} \\&=\sum_{j=0}^{n}\binom{n}{j} + \sum_{j=0}^{n}\binom{n}{j} = 2^n + 2^n = 2^{n+1}.
			\end{align*}
		\end{proof}
		\begin{proof}
			(alternative) \[ 2^n = (1 + 1)^n = \sum_{j=0}^{n}1^{n - j}1^j\binom{n}{j} = \sum_{j=0}^{n}\binom{n}{j} \]
		\end{proof}
		\item[(ii)] \[ \sum_{j=0}^{n}(-1)^j\binom{n}{j} = \binom{n}{0} - \binom{n}{1} + \dots \pm \binom{n}{n} = 0 \]
		\begin{proof}
			\[ 0 = (1 + (-1))^n = \sum_{j=0}^{n}1^{n-j}(-1)^j\binom{n}{j} = \sum_{j=0}^{n}(-1)^j\binom{n}{j}. \]
		\end{proof}
		\item[(iii)] \[ \sum_{l\text{ odd }}\binom{n}{l} = \binom{n}{1} + \binom{n}{3} + \dots = 2^{n-1} \]
		\begin{proof}
			\begin{align*}
				0 = \sum_{l=0}^{n}(-1)^l\binom{n}{l} = \sum_{l\text{ even }}\binom{n}{l} - \sum_{l\text{ odd }}\binom{n}{l},
			\end{align*} so \[ \sum_{l\text{ even }}\binom{n}{l} = \sum_{l\text{ odd }}\binom{n}{l}. \] Then
			\begin{align*}
				2^n = \sum_{l=0}^{m}\binom{n}{l} &= \sum_{l\text{ even }}\binom{n}{l} + \sum_{l\text{ odd }}\binom{n}{l} = 2\sum_{l\text{ odd }}\binom{n}{l},
			\end{align*} so \[ \sum_{l\text{ odd }} = \frac{2^n}{2} = 2^{n-1} \]
		\end{proof}
		\item[(iv)] \[ \sum_{l\text{ even }}\binom{n}{l} = \binom{n}{0} + \binom{n}{2} + \dots = 2^{n-1} \]
		\begin{proof}
			It follows from proof of (iii).
		\end{proof}
	\end{enumerate}
\end{enumerate}

\subsubsection*{5.}
\begin{enumerate}
	\item[(a)] Prove by induction on $n$ that \[ 1 + r + r^2 + \dots + r^n = \frac{1 - r^{n+1}}{1 - r} \] if $r \ne 1$.
	\begin{proof}
		Let $n = 1$. Then $1 + r = \frac{(1 + r)(1 - r)}{1 - r} = \frac{1 - r^2}{1 - r}$, so the formula holds.

		Next, suppose that the formula is true for some $n \ge 1$. Then
		\begin{align*}
			1 + r + r^2 + \dots + r^{n+1} &= \frac{1 - r^{n+1}}{1-r} + r^{n+1} = \frac{1 - r^{n+1}}{1-r} + \frac{r^{n+1}(1 - r)}{1 - r} \\&= \frac{1 - r^{n+1} + r^{n+1} - r^{n+2}}{1 - r} = \frac{1 - r^{n + 2}}{1 - r}
		\end{align*}
	\end{proof}
	\item[(b)] Derive this result by setting $S = 1 + r + \dots + r^n$, multiplying this equation by $r$, and solving the two equations for $S$.
	\begin{proof}
		\[ S = 1 + r + \dots + r^n \text{ and } Sr = r + r^2 + \dots + r^{n+1}. \] Then \[ S(1 - r) = 1 + r + \dots + r^n - r - r^2 - \dots - r^{n+1} = 1 - r^{n+1}, \] so \[ S = \frac{1 - r^{n+1}}{1 - r} \]
	\end{proof}21
\end{enumerate} 

\subsubsection*{6. The formula for $1^2 + \dots + n^2$ can be derived as follows. We begin with the formula \[(k + 1)^3 - k^3 = 3k^2 + 3k + 1.\] Writing this formula for $k = 1, \dots, n$ and adding, we obtain \[2^3 - 1^3 = 3 \cdot 1^2 + 3 \cdot 1 + 1\] \[ 3^3 - 2^3 = 3 \cdot 2^2 + 3 \cdot 2 + 1 \] \[\vdots\]\[ \frac{(n + 1)^3 - n^3 = 3n^2 + 3n + 1}{(n + 1)^3 - 1 = 4[1^2 + \dots + n^2] + 3[1 + \dots + n] + n.} \] Thus we can find $\sum_{k=1}^{n}k^2$ if we already know $\sum_{k=1}^{n}k$. Use this method to find}
\begin{enumerate}
	\item[(i)] $1^3 + \dots + n^3$.
	\begin{proof}
		We begin with \[ (k + 1)^4 - k^4 = 4k^3 + 6k^2 + 4k + 1 \]
		Then we have
		\begin{align*}
			(n + 1)^4 - 1 = 4\sum_{j=1}^{n}j^3 + 6\sum_{k=1}^{n}k^2 + 4\sum_{l=1}^{n}l + n
		\end{align*} so
		\begin{align*}
			\sum_{j=1}^{n}j^3 &= \frac{(n + 1)^4 - 1 - 6\sum_{k=1}^{n}k^2 - 4\sum_{l=1}^{n}l - n}{4} \\&= \frac{n^4 + 4n^3 + 6n^2 + 4n + 1 - 1 - 6\frac{n(n+1)(2n+1)}{6} - 4\frac{n(n+1)}{2} - n}{4} \\&= \frac{n^4 + 4n^3 + 6n^2 + 3n - 2n^3 - 3n^2 - n - 2n^2 -2n}{4} \\&= \frac{n^4}{4} + \frac{n^3}{2} + \frac{n^2}{4}
		\end{align*}
	\end{proof}
	\item[(ii)] $1^4 + \dots + n^4$.
	\begin{proof}
		We begin with \[ (k + 1)^5 - k^5 = 5k^4 + 10k^3 + 10k^2 + 5k + 1 \]
		Then we have
		\begin{align*}
			(n + 1)^5 - 1 = 5\sum_{i=1}^{n}i^4 + 10\sum_{j=1}^{n}j^3 + 10\sum_{k=1}^{n}k^2 + 5\sum_{l=1}^{n}l + n
		\end{align*} so
		\begin{align*}
			\sum_{i=1}^{n}i^4 &= \frac{(n + 1)^5 - 1 - 10\sum_{j=1}^{n}j^3 - 10\sum_{k=1}^{n}k^2 - 5\sum_{l=1}^{n}l - n}{5} \\&= \frac{n^5 + 5n^4 + 10n^3 + 10n^2 + 5n - 5\left(\frac{n^4}{4} + \frac{n^3}{2} + \frac{n^2}{4}\right) - 10\frac{n(n + 1)(2n + 1)}{6} - 5\frac{n(n+1)}{2} - n}{5} \\&= \frac{n^5}{5} + \frac{n^4}{2} + \frac{n^3}{3} - \frac{n}{30}
		\end{align*}
	\end{proof}
	\item[(iii)] $\frac{1}{1 \cdot 2} + \frac{1}{2 \cdot 3} + \dots + \frac{1}{n(n + 1)}$.
	\begin{proof}
		We begin with \[ \frac{1}{k} - \frac{1}{k + 1} = \frac{1}{k(k+1)} \] Then we have
		\begin{align*}
			\sum_{j=1}^{n}\frac{1}{n(n + 1)} = 1 - \frac{1}{n + 1}
		\end{align*}
	\end{proof}
	\item[(iv)] $\frac{3}{1^2 \cdot 2^2} + \frac{5}{2^2 \cdot 3^2} + \dots + \frac{2n + 1}{n^2(n + 1)^2}$.
	\begin{proof}
		We begin with \[ \frac{1}{(k+1)^2} - \frac{1}{k^2} = \frac{2k+1}{k^2(k+1)^2} \] Then we have
		\begin{align*}
			\sum_{j=1}^{n}\frac{2j+1}{j^2(j+1)^2} = \frac{1}{(n + 1)^2} - 1
		\end{align*}
	\end{proof}
\end{enumerate}

\subsubsection*{8. Prove that every natural number is either even or odd.}

\end{document}